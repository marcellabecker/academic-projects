\chapter{Desenvolvimento do projeto}
\label{chap:metod}
Nesta seção será descrito o procedimento utilizado para construção de cada um dos desafios.

\section{Webots}
O robô utilizado foi o Pionner o qual usa seus 16 sensores de distância pré-instalados para obter informações do mundo em todas as direções 
e um sensor adcional:o sensor de luminosidade que rastreia a irradiância local e envia um sinal para o robô quando ele lê mais de 750 W / m2, 
o que significa que está perto o suficiente da luminária de chão para disparar o STOP.

\subsection{controle}
A navegação do robô pelo mapa é baseada em uma máquina de 4 estados que determina se ele deve se mover para frente,
virar à esquerda, direita ou parar quando atingir seu objetivo final.

Assim, a divisão foi feita em quatro casos, são eles:
\begin{itemize}
    \item FORWARD: Anda para frente e se houver algum obstáculo à frente ele começa a tomar a decisão de virar em qualquer direção para evitá-lo.
    \item ESQUERDA: Vire à esquerda até que a detecção de objetos não seja mais possível;
    \item DIREITA: Vire à direita até que a detecção de objetos não seja mais possível;
    \item STOP: Quando o sensor de luz detecta a quantidade de luminosidade configurada (neste caso 750 W / m2), o robô deve parar.
\end{itemize}
Todos os sensores de distância coletam dados do ambiente que determinam se o robô deve se mover para frente ou para os lados.
\section{Turtlesim}
Para que o desafio seja cumprido é necessário rodar o roscore e o nó para que a janela da turtle apareça na tela.

\begin{figure} [h!]	
    \centering
    \caption{inicializando o ros}
    \includegraphics[width=0.5\textwidth]{roscore.jpg}
    \caption*{Fonte: Autoria própria.}
    \label{fig:roscore}
\end{figure}

\begin{figure} [h!]	
    \centering
    \caption{ Rodando o nó do turtlesim}
    \includegraphics[width=0.5\textwidth]{rosrum.jpg}
    \caption*{Fonte: Autoria própria.}
    \label{fig:rosrum}
\end{figure}
por fim executa-se o arquivo .py quue foi a liguagem escolhida para completar este desafio, como mostra a figura abaixo.
\begin{figure} [h!]	
    \centering
    \caption{ Executando arquivo}
    \includegraphics[width=0.5\textwidth]{xey.jpeg}
    \caption*{Fonte: Autoria própria.}
    \label{fig:excute.py}
\end{figure}
É posto as coordenadas x e y no caso do exemplo do desafio é 1 e 1 e a tartaruga deve ir até essa coordenada quando o erro for no máximo 0.1.
\section{Husky}
\subsection{Move Base}
Após toda a configuração e instalação dos pacotes do Husky encontrado no repositório https://github.com/husky/husky, foi executado o nó move-base no qual será enviado um comando para o robô Husky que tentará atingir a posição enviada desviando dos obstáculos e caso entre em alguma posição em que se esteja travado executa comportamentos de recuperação para continuar o trajeto enviado.

Para a execução do nó move-base é necessário trés comandos o primeiro inicia o ambiente de simulação o gazebo, o segundo o visualizador rviz e o terceiro a demonstração move-base.
\begin{figure} [h!]	
    \centering
    \caption{Comandos move base}
    \includegraphics[width=1.0\textwidth]{comandosmb.jpeg}
    \caption*{Fonte: Autoria própria.}
    \label{fig:movebase}
\end{figure}
\subsection{AMCL}
O Tutorial amcl do Husky mostra como é usado o move-base com o sendo assim, amcl obtém um mapa baseado em laser que precisa ser ativado no description do robô encontrado no próprio repositório após a ativação o laser faz varreduras e transforma em mensagens que geram estimativas de posição.
\begin{figure} [h!]	
    \centering
    \caption{Comandos amcl}
    \includegraphics[width=0.9\textwidth]{comandosamcl.jpeg}
    \caption*{Fonte: Autoria própria.}
    \label{fig:amcl}
\end{figure} 
\subsection{gmapping}
Após a execução do nó slam-gmapping será levado para o tópico sensor-msgs/LaserScan mensagens e constrói um mapa a partir dos dados de laser e posições coletados pelo husky.
\begin{figure} [h!]	
    \centering
    \caption{Comandos gmapping}
    \includegraphics[width=0.9\textwidth]{comandogm.jpeg}
    \caption*{Fonte: Autoria própria.}
    \label{fig:gmapping}
\end{figure}
\subsection{Frontier-exploration}
O Frontier-exploration pacote fornece um costmap-2d que fornece uma implementação de um mapa de custo 2D que leva em dados de sensor do mundo, constrói uma grade de ocupação 2D ou 3D dos dados, e actionlib cliente que fornece uma interface padronizada com tarefas preemptivas
\begin{figure} [h!]	
    \centering
    \caption{Comandos frontier-exploration }
    \includegraphics[width=1.0\textwidth]{comandosf.jpeg}
    \caption*{Fonte: Autoria própria.}
    \label{fig:frontier-exploration}
\end{figure}
\section{CPP}
Após assistir os tutoriais de CPP e obter os conhecimentos necessários foi possível resolver os desafios propostos no workbook de c++ os Desafios foram: 

    -Triângulo de Pascal que consiste em um programa para calcular o valor de uma determinada posição no Triângulo de Pascal.

    A maneira de calcular o valor de qualquer posição é somar os números à direita e à esquerda da posição na linha anterior.

   O programa solicita que o usuário insira uma linha e uma posição na linha e assim é dado o valor correspondente.

   Por exemplo, para calcular o número do meio na terceira linha, você adiciona 1 e 1; os lados do triângulo são sempre 1 porque você só adiciona o número à esquerda superior ou à direita superior, o exemplo pode ser melhor visualizado na figura abaixo.

\begin{figure} [h!]	
    \centering
    \caption{triangulo de pascal}
    \includegraphics[width=0.4\textwidth]{triangulodepascal.jpeg}
    \caption*{Fonte:https://www.todamateria.com.br/triangulo-de-pascal/}
    \label{fig:triangulodepascal}
\end{figure}


    -Desafio da Permutação de Cordas que se trata de um programa para exibir todas as permutações possíveis de uma determinada string de entrada.se a string contém caracteres duplicados, pode ter vários resultados repetidos.A saida deve ser dada uma palavra por linha.

\begin{figure} [h!]	
    \centering
    \caption{Permutação}
    \includegraphics[width=0.6\textwidth]{permutacao.jpeg}
    \caption*{Fonte:https://i.stack.imgur.com/8DK5W.gif}
    \label{fig:permutacaodeletras}
\end{figure}

    -Programa de Autoimpressão que é um programa que, quando executado, imprimirá seu código-fonte. Este código-fonte, por sua vez, deve compilar e imprimir a si mesmo. 
\section{Python}
Após assistir os tutoriais de Python e obter os conhecimentos necessários foi possível resolver os desafios propostos nos workbooks 1 e 2 de python os desafios foram: 

-Mediana de Três Valores que consiste em uma função que recebe três números como parâmetros e retorna o valor médio desses parâmetros como seu resultado.
\begin{figure} [h!]	
    \centering
    \caption{Mediana}
    \includegraphics[width=0.4\textwidth]{Achando_a_mediana.png}
    \caption*{Fonte:https://pt.wikipedia.org/wiki/Medianaestatistica}
    \label{fig:mediana}
\end{figure}

-Os Doze Dias do NatalOs Doze Dias de Natal é uma canção repetitiva que descreve uma lista cada vez mais longa de presentes enviados ao verdadeiro amor de cada um em cada um dos 12 dias. Um único presente é enviado no primeiro dia. Um novo presente é adicionado à coleção em cada dia adicional e, em seguida, a coleção completa é enviada.escrever um programa que exiba as letras completas de Os Doze Dias de Natal . Escreva uma função que tenha o número da estrofe como seu único parâmetro e exiba a estrofe especificada da música. Em seguida, chame essa função 12 vezes com números inteiros que aumentam de 1 a 12.

-Centralize uma corda no terminal Escreva uma função que tenha uma string de caracteres como seu primeiro parâmetro e a largura do terminal em caracteres como seu segundo parâmetro. Sua função deve retornar uma nova string que consiste na string original e o número correto de espaços iniciais para que a string original apareça centralizada dentro da largura fornecida quando for impressa. Não adicione nenhum caractere ao final da string. Inclua um programa principal que demonstre sua função.

-Capitalize-o muitas pessoas não usam letras maiúsculas corretamente, especialmente ao digitar em pequenos dispositivos como smartphones. Neste exercício, você escreverá uma função que coloca em maiúscula os caracteres apropriados em uma string.Um “i” minúsculo deve ser substituído por um “I” maiúsculo se for precedido e seguido por um espaço. O primeiro caractere na string também deve estar em maiúscula, assim como o primeiro caractere sem espaço após um “.”, “!” ou "?". Por exemplo, se a função for fornecida com a string “que horas eu tenho que estar lá? qual é o endereço?" então ele deve retornar a string “Que horas eu tenho que estar lá? Qual é o endereço?". Inclui um programa principal que lê uma string do usuário, coloca-a em maiúscula usando sua função e exibe o resultado.

-Uma string representa um inteiro uma função chamada isInteger que determina se os caracteres em uma string representam ou não um inteiro válido. Ao determinar se uma string representa um inteiro, você deve ignorar qualquer espaço em branco à esquerda ou à direita. Uma vez que este espaço em branco é ignorado, uma string representa um inteiro se seu comprimento for pelo menos 1 e contiver apenas dígitos, ou se seu primeiro caractere for + ou - e o primeiro caractere for seguido por um ou mais caracteres, todos os quais são dígitos.

-É um número primo?Um número primo é um número inteiro maior que 1 que só é divisível por um e por ele mesmo. Escreva uma função que determine se seu parâmetro é primo ou não, retornando True se for, e False caso contrário. 

-senha aleatória Escreva uma função que gere uma senha aleatória. A senha deve ter um comprimento aleatório de 7 a 10 caracteres. Cada caractere deve ser selecionado aleatoriamente das posições 33 a 126 na tabela ASCII. Sua função não terá nenhum parâmetro. Ele retornará a senha gerada aleatoriamente como seu único resultado.

-Verifique uma senha Neste exercício, você escreverá uma função que determina se uma senha é válida ou não. Definiremos uma boa senha como aquela que tem pelo menos 8 caracteres e contém pelo menos uma letra maiúscula, pelo menos uma letra minúscula e pelo menos um número. Sua função deve retornar verdadeiro se a senha passada a ela como seu único parâmetro for válida.

-Conversões de base arbitrária Escreva um programa que permita ao usuário converter um número de uma base para outra. Seu programa deve oferecer suporte a bases entre 2 e 16, tanto para o número de entrada quanto para o número de resultado. Se o usuário escolher uma base fora dessa faixa, uma mensagem de erro apropriada deve ser exibida e o programa deve ser encerrado.

-Reduzir uma fração aos termos mais baixos Escreva uma função que recebe dois inteiros positivos que representam o numerador e o denominador de uma fração como seus únicos dois parâmetros. 

-Reduzir Medidas Escreva uma função que expresse um volume imperial usando as maiores unidades possíveis. A função terá o número de unidades como seu primeiro parâmetro e a unidade de medida (xícara, colher de sopa ou colher de chá) como seu segundo parâmetro. Retorne uma string representando a medida usando as maiores unidades possíveis como o único resultado da função. 

-Datas mágicas Uma data mágica é uma data em que o dia multiplicado pelo mês é igual ao
ano de dois dígitos . Por exemplo, 10 de junho de 1960 é uma data mágica porque junho é o sexto mês e
6 vezes 10 é 60, que é igual ao ano de dois dígitos. Escreva uma função que determina
se uma data é ou não uma data mágica. Use sua função para criar um programa principal
que encontra e exibe todas as datas mágicas do século XX.
% %--------- NEW SECTION ----------------------
% \section{Interface do Usuário}
% \label{sec:ui}
% \lipsum[1]

% %--------- NEW SECTION ----------------------
% \section{Simulação do sistema}
% \label{sec:sim}
% \lipsum[2-4]

