\chapter{Introdução}
\label{chap:intro}
\subsection{Webots}
Webots é um aplicativo de desktop de código aberto e multiplataforma usado para simular robôs. 
Ele fornece um ambiente de desenvolvimento completo para modelar, programar e simular robôs.
Ele foi projetado para um uso profissional e é amplamente utilizado na indústria, educação e pesquisa.
Cyberbotics Ltd. mantém Webots como seu produto principal continuamente desde 1998.
\subsection{Turtlesim SETPOINT POSITION }
Uma maneira simples de aprender o básico de ROS é usando o simulador turtlesim. Que consiste na simulação 
de uma janela grafica que mostra um robô com formato de tartaruga. Esta tartaruga pode ser movidar por 
toda a janela utilizando comandos de ROS como roscore, rosrum... ou utilizando o teclado/joystick.
\subsection{Husky}
O Husky é um veículo terrestre não tripulado (UGV) robusto, pronto para uso ao ar livre, adequado para 
pesquisas e aplicações de prototipagem rápida e oferece suporte total a ROS.
\subsection{CPP workbook}
O C++ é uma linguagem de programação de nível médio, baseada na linguagem C.
Tem uma enorme variedade de códigos, pois alem de seus códigos, pode contar com vários da linguagem C.
Esta variedade possibilita a programação em alto e baixo níveis.
\subsection{Python workbook1 e 2}
Python é uma linguagem de programação de alto nível, ou seja, com sintaxe mais simplificada e próxima da 
linguagem humana, utilizada nas mais diversas aplicações, como desktop, web, servidores e ciência de dados.
%--------- NEW SECTION ----------------------
\section{Objetivos}
\label{sec:obj}
\begin{itemize}
      \item Webots:
Desenvolver um sistema de navegação autônoma, de forma que o robô consiga chegar 
á região iluminada do mapa pré-definido, evitando todos os obstáculos do percurso em 2 minutos.
      \item Turtlesim:
Fazer um controlador de posição para a tartaruga que da um ponto (X, Y) a tartaruga 
deve atingir o ponto com certo erro associado.
      \item Husky:
Simular a navegação do robô Husky como está descrito no tutorial do ROS.
      \item CPP workbook:
Demonstrar os conhecimentos adquiridos nos tutoriais de c++.
      \item Python workbook1 e 2:
Fornecer uma oportunidade para que todos evoluam no conceito de codificação.
  \end{itemize}
\subsection{Objetivos Específicos}
\label{ssec:objesp}
\subsection{Webots}
Os objetivos específicos deste desafio são:
\begin{itemize}
      \item Fazer navegação do PIONEER;
      \item Utilizar o repositório https://github.com/Brazilian-Institute-of-Robotics/desafiorobotica.git;
      \item Utilizar sensor de luminosidade;
      \item Simular no webots;
  \end{itemize}
\subsection{Turtlesim SETPOINT POSITION}
Os objetivos específicos deste desafio são:
\begin{itemize}
      \item A posição do ponto de ajuste deve ser definida por meio de linha de comando;
      \item A escolha da linguagem de programação é Python;
      \item O código deve estar disponível em um repositório GitHub;
      \item O erro aceito é em torno de 0,1;
  \end{itemize}
\subsection{Husky}
Os objetivos específicos deste desafio são:
\begin{itemize}
      \item Fazer o tutorial Husky;
      \item Utilizar os pacotes do Husky;
      \item Simular a navegação nos 4 modos;
      \item Simular usando o gazebo;
  \end{itemize}
\subsection{CPP workbook}
Os objetivos específicos deste desafio são:
\begin{itemize}
      \item Fazer os desafios propostos;
      \item Utilizar C++ para a resolução;
      \item O código deve estar disponível em um repositório GitHub;
  \end{itemize}
  \subsection{Python workbook1 e 2}
  Os objetivos específicos deste desafio são:
  \begin{itemize}
        \item Fazer os desafios propostos;
        \item Utilizar Python para a resolução;
        \item Cada exercício deve ser preparado na menor quantidade de linhas possível;
        \item Os comentários são necessários para você explicar cada etapa do código;
    \end{itemize}
%--------- NEW SECTION ----------------------
\section{Justificativa}
\label{sec:justi}

Demonstrar os conhecimentos adquiridos dos topicos de Webots,ROS,CPP e python durante o processo de todos os desafios, 
a fim de ter um ótimo desempenho nas atividades do Laboratório.

%--------- NEW SECTION ----------------------
\section{Organização do documento}
\label{section:organizacao}

Este documento apresenta $5$ capítulos e está estruturado da seguinte forma:

\begin{itemize}

  \item \textbf{Capítulo 1 - Introdução}: Contextualiza o âmbito, no qual os desafios estão inseridos. Apresenta, portanto, a definição dos desafios, objetivos e justificativas do mesmo, alem de como este \thetypeworkthree está estruturado;
  \item \textbf{Capítulo 2 - Fundamentação Teórica}: As teorias usadas em cada um dos desafios;
  \item \textbf{Capítulo 3 - Materiais e Métodos}: De que forma os desafios foram feitos;
  \item \textbf{Capítulo 4 - Resultados}: Os resultados obtidos em cada um dos desafios propostos;
  \item \textbf{Capítulo 5 - Conclusão}: Apresenta as conclusóes.

\end{itemize}
